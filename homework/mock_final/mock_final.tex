 \documentclass[10pt]{article}
\usepackage{fullpage}
\usepackage{color}
\usepackage{amssymb}
\title{\color{blue}\bf Math 308: Last Homework Assignment}
\date{Wednesday, May 31, 2013}
\author{by William Stein}
\newcommand{\tf}[1]{\item {\bf {\color{blue}\hspace{1em}T\hspace{1em}F}}\hspace{1em} #1}
\newcommand{\R}{\mathbf{R}}
\newcommand{\C}{\mathbf{C}}
\newcommand{\F}{\mathbf{F}}
\renewcommand{\P}{\mathcal{P}}
\begin{document}

\maketitle
{\noindent\bf\color{red} Instructions:}
\begin{itemize}
\item {\bf Due on midnight, Wednesday, June 5 via this page:} \url{http://goo.gl/rd8zi}
\item The final exam will be similar to this, but longer.
\end{itemize}

{\noindent\bf \color{red} Questions (circle T or F):}
\begin{enumerate}

\tf{A system of linear equations over $\R$ has either no solutions or infinitely many solutions.}

\tf{The span of the rows of the matrix $A=\left(\begin{array}{rrr}
1 & 2 & 3 \\
4 & 5 & 6 \\
7 & 8 & 9
\end{array}\right)$ has dimension 3.}

\tf{There are 4 linearly independent vectors in $\R^5$.}

\tf{The upper half plane, i.e., the set of points $(x,y) \in \R^2$ with $y\geq 0$, is a subspace of $\R^2$.}

\tf{If $A$ and $B$ are any two square matrices (of the same size), then $\det(AB)=\det(BA)$ and $\det(A+B)=\det(B+A)$.}

\tf{Let $\F_3$ be the finite field with $3$ elements and let $V$ be a $2$-dimensional
    vector space over $\F_3$.  Then there are 81 linear transformations $V\to V$.}

\tf{If $A$ is any $n\times n$ matrix and $B$ is the reduced row echelon form of $A$, then $\det(A)=\det(B^t)$, where
$B^t$ is the transpose of $B$.}

\tf{Suppose $A$ is a $3 \times 3$ matrix with eigenvalues $1,2,3$. Then $A$ must be diagonalizable.}

\tf{Suppose $A$ is a $3 \times 3$ matrix with eigenvalues $1$ and $-1$.  Then $A^2=I_3$.}

\tf{Let $\P$ be the vector space over $\R$ of polynomials of degree at most $3$ with real coefficients,
and let $T:\P\to\P$ be the linear transformation $T(ax^2+bx+c)=a+b+c$.  Then there is a basis $\mathcal{B}$
for $\P$ such that $[T]_{\P,\P}$ is diagonal.}

\tf{Let $\P$ be the vector space over $\R$ of polynomials of degree at most $3$ with real coefficients,
and let $T:\P\to\R^1$ be the linear transformation $T(ax^2+bx+c)=a+b+c$.
Then the kernel of $T$ (i.e., the set of $v$ with $T(v)=0$) has dimension $1$.}

\tf{Suppose $V$ and $W$ are vector spaces over $\R$ of dimensions $2$ and $3$, respectively.
Then there is a linear transformation $T:V\to W$ such that $\ker(T)=0$.}

\tf{Suppose $V$ and $W$ are vector spaces over $\R$ of dimensions $3$ and $2$, respectively.
Then there is a linear transformation $T:V\to W$ such that $\ker(T)=0$.}

\tf{Every $2\times 2$ matrix $A$ is diagonalizable (over the complex numbers $\C$).}

\tf{The matrix $A=\left(\begin{array}{rr}
1 & 2 \\
3 & 4
\end{array}\right)$ is diagonalizable over $\C$.}

\tf{Let $V$ be the vector space of all differentiable functions $\R\to \R$.
The function that send $f$ to its derivative is a linear transformation $V\to V$.}

\tf{The dimension of the kernel of the derivative transformation (defined in the previous problem) is $1$.}

\tf{Let $V$ be the set of all integrable functions $\R\to \R$, i.e., functions that
have some antiderivative.  Then $V$ is a vector space.}

\tf{Let $V$ be the set of all integrable functions $\R\to \R$.
The function that $f$ to ``the antiderivative of $f$ that sends
$0$ to $0$'' is a linear transformation of $V$.}

\tf{Let $V$ be the set of all integrable functions $\R\to \R$.
The function that $f$ to ``the antiderivative of $f$ that sends
$0$ to $1$'' is a linear transformation of $V$.}

\tf{The determinant of $A=\left(\begin{array}{rrrr}
1 & 2 & 3 & 4 \\
5 & 6 & 7 & 8 \\
9 & 10 & 11 & 12 \\
13 & 14 & 15 & 16
\end{array}\right)$ is $0$.}

\tf{If $B$ and $C$ are basis for a (finite dimensional) vector space $V$, then there must be some linear transformation
$T:V\to V$ such that $[T]_{B,C}$ is the identity matrix.}

\tf{Let $V$ be the set of functions $\R\to\R$.  Then the function $V\to \R^2$ that sends $f$ to $(f(1), f(\pi))$
is a linear transformation.}

\tf{Let $V$ be the vector space over $\R$ that is the span of $\sin(x)$, $\cos(x)$, and $\cos(3x)$.
Let $T:V\to \R^1$ be the linear transformation that sends $f\in V$ to $f(0)$.
Then the kernel of $T$ has dimension $1$.}

\tf{Let $V$ be the vector space over $\R$ that is the span of $\sin(x)$, $\cos(x)$, and $\cos(3x)$.
Let $T:V\to \R^1$ be the linear transformation that sends $f\in V$ to $f(0)$.
Then $T$ is surjective, i.e., for every $a\in \R^1$ there is $v\in V$ such that $T(v)=a$.}

\tf{Suppose $A$ is a matrix with characteristic polynomial $x(x-1)(x-2)(x-3)$. Then
there is a basis $\mathcal{B}$ of $\R^4$ such that
$[A]_{\mathcal{B},\mathcal{B}} = \left(\begin{array}{rrrr}
3 & 0 & 0 & 0 \\
0 & 1 & 0 & 0 \\
0 & 0 & 2 & 0 \\
0 & 0 & 0 & 0
\end{array}\right).$}

\tf{Suppose $A$ has characteristic polynomial $x^{7} + x^3 - 3$.  Then $\det(A)\neq 0$.}

\end{enumerate}

\end{document}