\documentclass[12pt]{article}
\usepackage{fullpage}
\usepackage{color}
\usepackage{url}
\usepackage{amssymb}
\usepackage{amsmath}
\title{\color{blue}\bf Math 308: Final Examination}
\date{Wednesday, June 12, 2013, 2:30--4:20 p.m.}
\author{by William Stein}
\newcommand{\tf}[1]{\item {\bf {\color{blue}\hspace{1em}T\hspace{1em}F}}\hspace{1em} #1\vspace{1.1ex}}
\newcommand{\R}{\mathbb{R}}
\newcommand{\C}{\mathbb{C}}
\newcommand{\F}{\mathbb{F}}
\renewcommand{\P}{\mathcal{P}}
\begin{document}

\maketitle
{\noindent\bf\color{red} Instructions:}
\begin{itemize}
\item {\color{blue} Your Name: \underline{\hspace{20em}}\\ID Number: \underline{\hspace{15em}}}

\item Exactly as with the midterms, you may use one sheet of paper with notes on it, and you may use a calculator.
\item Turn in your exam at 4:20pm, so you have 1:50 to complete this exam.  Answer every question.  Triple check your work!  $\square\quad\square\quad\square$
\end{itemize}

{\noindent\bf \color{red} Questions (circle T or F):}
\begin{enumerate}

\tf{The following $2\times 5$ matrix is in reduced row echelon form:
$$\left(\begin{array}{rrrrr}
0 & 1 & 2 & 0 & 0\\
0 & 0 & 0 & 1 & 2
\end{array}\right)
$$
}

\tf{The following $1\times 6$ matrix is in reduced row echelon form:
$$\left(\begin{array}{rrrrrr}
0 & 1 & 2 & 3 & 4 & 5\\
\end{array}\right)
$$
}

\tf{The span of the rows of the matrix $A=\left(\begin{array}{rrr}
1 & 0 & 3 \\
4 & 0 & 6 \\
7 & 0 & 9
\end{array}\right)$ has dimension 2.}

\tf{The span of the rows of the matrix $A=\left(\begin{array}{rrr}
1 & 2 & 3 \\
4 & 5 & 6 \\
7 & 8 & 9
\end{array}\right)$ has dimension 3.}

\tf{The set of $\left(\begin{array}{c}
x \\
y\\
z
\end{array}\right)\in \R^3$ such that
\begin{align*}
x + 2 \, y + 3 \, z &= 1 \\
4 \, x + 5 \, y + 6 \, z &= 2\\
7 \, x + 8 \, y + 9 \, z &= 3
\end{align*}
is a vector space of dimension $1$.}

\tf{The set of $\left(\begin{array}{c}
x \\
y\\
z
\end{array}\right)\in \R^3$ such that
\begin{align*}
x + 2 \, y + 3 \, z &= 0 \\
4 \, x + 5 \, y + 6 \, z &= 0\\
7 \, x + 8 \, y + 9 \, z &= 0
\end{align*}
is a vector space of dimension $2$.}

\tf{The lower half plane, i.e., the set of points $(x,y) \in \R^2$ with $y\leq 0$, is not a subspace of $\R^2$.}

\tf{If $A$ and $B$ are any two square matrices (of the same size), then $(AB)^T=B^T A^T$ and $(A+B)^T= B^T + A^T$.}

\tf{If $A$ and $B$ are any two invertible square matrices (of the same size) such that $A+B$ is also invertible,
then $(AB)^{-1}=B^{-1} A^{-1}$ and $(A+B)^{-1}= B^{-1} + A^{-1}$.}

\tf{Let $\F_2$ be the finite field with $2$ elements and let $V$ be a $2$-dimensional
    vector space over $\F_2$. (In the book $\F_2$ is denoted $\mathbb{Z}_2$.)   Then there are exactly 16 distinct linear transformations $V\to V$.}


\tf{If $A$ is any $n\times n$ matrix and $B$ is the reduced row echelon form of $A$, then $\det(A)=0$ if and only if $\det(B)=0$.}

\tf{If $A$ is any $n\times n$ matrix and $B$ is the reduced row echelon form of $A$, then $\det(A)=1$ if and only if $\det(B)=1$.}

\tf{Let $\P_3$ be the vector space over $\R$ of polynomials of degree at most $3$ with real coefficients,
and let $T:\P_3\to\P_3$ be the linear transformation $T(ax^2+bx+c)=cx + b-a$.  Then there is a basis $\mathcal{B}$ for $\P_3$ such that $[T]_{\mathcal{B},\mathcal{B}}$ is diagonal.}

\tf{Let $\P_3$ be the vector space over $\R$ of polynomials of degree at most $3$ with real coefficients,
and let $T:\P_3\to\P_3$ be the linear transformation $T(ax^2+bx+c)=cx+b-a$.
Then the kernel of $T$ (i.e., the set of $v$ with $T(v)=0$) has dimension $1$.}

\tf{Let $\P_2$ be the vector space over $\R$ of polynomials of degree at most $2$ with real coefficients,
and let $T:\P_2\to\P_2$ be the linear transformation $T(bx+c)=cx-b$.  Then the matrix $[T]_B$ of $T$
with respect to the basis $1,x$ is $\left(\begin{array}{rr}
0 & 1 \\
-1 & 0
\end{array}\right)$.
}


\tf{The matrix $A=\left(\begin{array}{rr}
1 & 2013 \\
0 & 1
\end{array}\right)$ is diagonalizable over $\C$.}

\tf{The subset ${\rm GL}_3(\C)$ of all invertible $3\times 3$ complex matrices is a vector subspace of
the vector space $M_{3,3}(\C)$ of all $3\times 3$ complex matrices.}

\tf{The matrix $A=\left(\begin{array}{rr}
1 & 2013 \\
2013 & 1
\end{array}\right)$ is diagonalizable over $\C$.}

\tf{Let $V$ be the vector space of all infinitely differentiable functions $\R\to \R$.
The function that send $f$ to its third derivative is a linear transformation $V\to V$ whose
kernel has dimension $2$.}

\tf{Let $V$ be the vector space of all infinitely differentiable functions $\R\to \R$.
The kernel of the linear transformation that sends $f$ to its third derivative is
the space $\mathcal{P}_{2}$ of polynomials of degree at most~$2$.}

\tf{The determinant of $A$ is $0$, where $$A=\left(\begin{array}{rrrr}
-\frac{12825}{11504} & \frac{90625}{88467} & \frac{49455}{35626} & -\frac{6151}{38520}\vspace{1ex}\\
\frac{48955}{10471} & \frac{60835}{89249} & \frac{17564}{97411} & -\frac{7103}{3700} \vspace{1ex}\\
-\frac{12825}{11504} & \frac{90625}{88467} & \frac{49455}{35626} & -\frac{6151}{38520}\vspace{1ex}\\
-\frac{2275}{7108} & \frac{49132}{79339} & -\frac{49791}{670} & \frac{6115}{4868}
\end{array}\right).$$}

\tf{Let $V$ be the vector space of all functions $\R\to\R$.  Then the function $V\to \R^2$ that sends $f$ to $(0, f(0))$
is a linear transformation.}

\tf{Let $V$ be the vector space of all functions $\R\to\R$.  Then the function $V\to \R^5$ that sends $f$ to $(f(-1),f(1),f(2),f(\pi),f(4))$
is a linear transformation.}

\tf{There are infinitely many distinct orthonormal bases for~$\R^2$.}

\tf{There is exactly one orthonormal basis for $\R^1$.}

\tf{If $A$ is any orthogonal matrix, then $\det(A)=1$.}

\tf{If $A$ is any $2\times 2$ matrix with real entries such that $\det(A)=1$, then $A$ must be an orthogonal matrix.}

\tf{If $A$ is any $2\times 2$ matrix with real entries such that $A^2=I_2$, then $A$ must be an orthogonal matrix.}

\tf{Suppose $A$ is a $2\times 2$ matrix with  characteristic polynomial $x^{2} + 3x$.  Then $A$ is not invertible.}

\tf{The vectors $v_1=\left(\begin{array}{rrr}
1 \\
2  \\
0\\
-1
\end{array}\right)$,
$v_2=\left(\begin{array}{rrr}
2\\
1\\
1\\
4
\end{array}\right)$  are an orthogonal basis for a subspace of $\R^4$.}

\tf{Suppose $A$ is a $3\times 5$ matrix of rank $2$ (with real entries).  Then the dimension of the nullspace of $A$ must be $3$.}

\tf{Suppose $A$ is a $3\times 5$ matrix with real entries that looks like this (where the entries marked with * can be absolutely anything): $A={\left(\begin{array}{rrrrr}
1 & 2 & 1 & * & * \\
-1 & 1 & 1 & * & * \\
2 & 0 & 1 & * & *
\end{array}\right)}$.  Then the dimension of the nullspace of $A$ must be $3$.}

\tf{Suppose $A$ is any orthogonal $3\times 3$ matrix.  Then $A$ must be in reduced row echelon form.}

\tf{Suppose $A$ is any orthogonal $3\times 3$ matrix.  Then $A=A^T$.}

\tf{Suppose $A$ is any diagonalizable $3\times 3$ matrix over the complex numbers with eigenvalues $a,b,c$ and corresponding eigenvectors $v_1, v_2, v_3$.  Let $B$ be the matrix with columns $v_1, v_2, v_3$ and $D$ the diagonal matrix with entries $a,b,c$.  Then $$D^{-1} = B^{-1} A^{-1} B.$$}

\tf{Let $V$ be a finite dimensional vector space, $b_1,\ldots, b_n$ a basis $B$ for $V$
and $c_1,\ldots, c_n$ a basis $C$ for $V$.   Let $T$ be the unique linear transformation
such that $T(b_i)=c_i$ for each $i$.  Then $[T]_{B,C} = I_n$ is the identity matrix.}

\end{enumerate}

\end{document}